%% LyX 2.1.4 created this file.  For more info, see http://www.lyx.org/.
%% Do not edit unless you really know what you are doing.
\documentclass[english]{article}
\usepackage[T1]{fontenc}
\usepackage[latin9]{inputenc}
\usepackage{amsmath}

\makeatletter
%%%%%%%%%%%%%%%%%%%%%%%%%%%%%% User specified LaTeX commands.
\usepackage[margin=0.75in]{geometry} % see geometry.pdf on how to lay out the page. There's lots.
\usepackage{graphicx}
\usepackage{cleveref}
\usepackage{amsmath}
\usepackage{multirow}
\usepackage{listings}
\usepackage{color}
\usepackage{CJK}
\definecolor{mygreen}{RGB}{28,172,0}
\definecolor{mylilas}{RGB}{170,55,241}

\usepackage[latin9]{inputenc}
\usepackage{geometry}
\geometry{verbose}


\makeatletter
\@ifundefined{date}{}{\date{}}
\makeatother

%Fancy-header package to modify header/page numbering 
\usepackage{fancyhdr}
\pagestyle{fancy}
\lhead{\textbf{ESE 101}} %name of the course
\chead{\textbf{Kangchen Bai}} %topic of the homework set
\rhead{\textbf{HW 8}} %number of the homework set
\lfoot{}
\cfoot{}
\rfoot{\thepage}


% Matlab script
\lstset{language=Matlab,%
      %basicstyle=\color{red},
  breaklines=true,%
  morekeywords={matlab2tikz},
  keywordstyle=\color{blue},%
  morekeywords=[2]{1}, keywordstyle=[2]{\color{black}},
  identifierstyle=\color{black},%}
  stringstyle=\color{mylilas},
  commentstyle=\color{mygreen},%
  showstringspaces=false,%without this there will be a symbol in the places where there is a space
  numbers=left,%
  numberstyle={\tiny \color{black}},% size of the numbers
  numbersep=9pt, % this defines how far the numbers are from the text
  emph=[1]{for,end,break},emphstyle=[1]\color{red}, %some words to emphasise
                                                      %emph=[2]{word1,word2}, emphstyle=[2]{style},    
}

\makeatother

\usepackage{babel}
\begin{document}

\subsection*{1}


\subsubsection*{(a)}

incompressible of water:

$\nabla\cdotp x=\frac{\partial u}{\partial x}+\frac{\partial v}{\partial y}+\frac{\partial w}{\partial z}=0$

integrate along $z$

we have $0=\frac{Dh}{Dt}+h(\nabla\cdotp v)$

Since $\nabla\cdotp(hv)=\nabla h\cdotp v+h(\nabla\cdotp v)$

and $\frac{Dh}{Dt}=\frac{\partial h}{\partial t}+\nabla h\cdotp v$

so $\frac{\partial h}{\partial t}+\nabla\cdotp(hv)=\frac{Dh}{Dt}+h(\nabla\cdotp v)=0$


\subsubsection*{(b)}

Because $p=p_{a}+\rho g[h(x,y,t)-z]$

so $\frac{1}{\rho}(\nabla p)=\frac{1}{\rho}(\nabla(\rho g[h(x,y,t)-z]))$

If we assume $z=h=const$ and $\rho$ is constant (because we assume
water to be incompressible)

$-\frac{1}{\rho}(\nabla p)=\frac{1}{\rho}(\nabla(\rho g[h(x,y,t)-z]))=-g\nabla h$

Besides this pressure force, the water also bear the coriolis force:

$-f\hat{k}\times v$

$a=-f\hat{k}\times v-g\nabla h$

and $a=\frac{Dv}{Dt}$

so $\frac{Dv}{Dt}=\frac{\partial v}{\partial t}+v\cdotp(\nabla v)$
here $\nabla v$ is a 2 order tensor $\begin{array}{cc}
\frac{\partial u}{\partial x} & \frac{\partial u}{\partial y}\\
\frac{\partial v}{\partial x} & \frac{\partial v}{\partial y}
\end{array}$

so $\frac{Dv}{Dt}=\frac{\partial v}{\partial t}+u\frac{\partial v}{\partial x}+v\frac{\partial v}{\partial y}$

so we have equation (3)


\subsubsection*{(c)}

$u\frac{\partial v}{\partial x}+v\frac{\partial v}{\partial y}=\hat{i}(uu_{x}+vu_{y})+\hat{j}(uv_{x}+vv_{y})$


\subsection*{2}


\subsubsection*{(a)}

$\frac{\partial u}{\partial t}+u\frac{\partial u}{\partial x}+v\frac{\partial u}{\partial y}=-g\frac{\partial h}{\partial x}$
(1)

$\frac{\partial v}{\partial t}+u\frac{\partial v}{\partial x}+v\frac{\partial v}{\partial y}=-g\frac{\partial h}{\partial y}$
(2)

neglecting those $v\cdotp\nabla v$ terms in (1) and (2)

$\frac{\partial^{2}u}{\partial x\partial t}=-g\frac{\partial^{2}h}{\partial x^{2}}$
and $\frac{\partial^{2}v}{\partial y\partial t}=-g\frac{\partial^{2}h}{\partial y^{2}}$

so $\frac{\partial^{2}u}{\partial x\partial t}+\frac{\partial^{2}v}{\partial y\partial t}=-g(\frac{\partial^{2}h}{\partial x^{2}}+\frac{\partial^{2}h}{\partial y^{2}})$

and from (1) we know that

$h(\frac{\partial u}{\partial x}+\frac{\partial v}{\partial y})=-\frac{\partial h}{\partial t}$

so 

$\frac{\partial^{2}u}{\partial x\partial t}+\frac{\partial^{2}v}{\partial y\partial t}=\frac{\partial}{\partial t}(\frac{\partial u}{\partial x}+\frac{\partial v}{\partial y})=-\frac{1}{h}\frac{\partial^{2}h}{\partial t^{2}}$

so $\frac{1}{h}\frac{\partial^{2}h}{\partial t^{2}}=g(\frac{\partial^{2}h}{\partial x^{2}}+\frac{\partial^{2}h}{\partial y^{2}})=g\nabla^{2}h$

and the tsunami wave speed is thus $\sqrt{gh_{0}}$


\subsubsection*{(b)}

The wave number is $K=(k,m)$

The frequency of the wave is $\omega$

so the wave speed is $\frac{\omega}{|K|}=\frac{\omega}{\sqrt{m^{2}+k^{2}}}$

for average ocean depth of 4km

$c=\sqrt{gh_{0}}=\sqrt{9.8*4000}=197.98m/s$

for average thickness of troposphere

$c=\sqrt{gh_{0}}=\sqrt{9.8*9000}=296.98m/s$


\section*{3}


\subsection*{(a)}

The governing equations are now:

$\frac{\partial u}{\partial t}+fv+g\frac{\partial h}{\partial x}=0$

$\frac{\partial v}{\partial t}-fu+g\frac{\partial h}{\partial y}=0$

$\frac{\partial h}{\partial t}+h(\frac{\partial u}{\partial x}+\frac{\partial v}{\partial y})=0$

If we assume the form

$u=u_{0}exp(ikx+imy-i\omega t)$

$v=v_{0}exp(ikx+imy-i\omega t)$

$h=h_{0}exp(ikx+imy-i\omega t)$

then

$-i\omega u_{0}+fv_{0}+igkh_{0}=0$

$-i\omega v_{0}-fu_{0}+igmh_{0}=0$

$-i\omega h_{0}+iHku_{0}+iHmv_{0}=0$

so det$\begin{bmatrix}-i\omega & f & igk\\
-f & -i\omega & igm\\
iHk & iHm & -i\omega
\end{bmatrix}=0$

so $\omega^{3}-\omega(gH(k^{2}+m^{2})+f^{2})=0$

so $\omega=0,\sqrt{gH(k^{2}+m^{2})+f^{2}},-\sqrt{gH(k^{2}+m^{2})+f^{2}}$

Because $c=\frac{\omega}{\sqrt{m^{2}+k^{2}}}=\sqrt{gH+f^{2}/(k^{2}+m^{2})}$

so it propagates faster.


\subsubsection*{(c)}

Because if $\omega=0$ then the determinant must be 0.

So the three equations must be linearly dependent.

that is 

$fv+g\frac{\partial h}{\partial x}=0$

$-fu+g\frac{\partial h}{\partial y}$

$\frac{\partial u}{\partial x}+\frac{\partial v}{\partial y}=0$

That is just the geostrophic balance.


\subsection*{4}

$\frac{\partial u}{\partial t}+u\frac{\partial u}{\partial x}+v\frac{\partial u}{\partial y}=fv-g\frac{\partial h}{\partial x}$
(1)

$\frac{\partial v}{\partial t}+u\frac{\partial v}{\partial x}+v\frac{\partial v}{\partial y}=-fu-g\frac{\partial h}{\partial y}$
(2)

$u_{yt}+u_{y}u_{x}+uu_{xy}+v_{y}u_{y}+vu_{yy}=fv_{y}-gh_{xy}+vf_{y}$(3)

$v_{xt}+u_{x}v_{x}+uv_{xx}+v_{x}v_{y}+vv_{xy}=-fu_{x}-gh_{xy}$(4)

(3)-(4)

$u_{yt}-v_{xt}+u_{y}u_{x}+v_{y}u_{y}-u_{x}v_{x}-v_{x}v_{y}+uu_{xy}+vu_{yy}-uv_{xx}-vv_{xy}$

$=fv_{y}+fu_{x}$

taking in $\xi=(v_{x}-u_{y})$

$-(u\xi_{x}+v\xi_{y})-\xi(\nabla\cdotp V)=f(\nabla\cdotp V)+vf_{y}$

$V\cdotp\nabla\xi+(\xi+f)(\nabla\cdotp V)+vf_{y}=0$

$\frac{D\xi}{Dt}=\frac{\partial\xi}{\partial t}+V\cdot\nabla\xi$

If we assume a stationary solution , then $\frac{\partial\xi}{\partial t}=0$

so we have $\frac{D\xi}{Dt}+(\xi+f)(\nabla\cdotp V)+vf_{y}=0$

Because $vf_{y}=\frac{Df}{Dt}$

so $\frac{D(\xi+f)}{Dt}+(\xi+f)\delta=0$

combing with eq (1)

$\frac{DQ}{Dt}=D(\frac{\xi+f}{h})/Dt=\frac{1}{h^{2}}(\frac{D(\xi+f)}{Dt}h-\frac{Dh}{Dt}(\xi+f))=\frac{1}{h^{2}}(-(\xi+f)\delta h+(h\delta)(\xi+f))=0$


\subsection*{5}


\subsubsection*{(a)}

The leading terms in eq(1)

$h_{t}+h(u_{x}+v_{y})=0$

leading terms in eq(3)

$v_{t}+f\hat{k}\times v+g\nabla h=0$

if we also neglect the time dependent term:

$-fv+gh_{x}=0$(a)

and 

$fu+gh_{y}=0$(b)

if we take $y$ derivative of (a) minus $x$ derivative of (b)

then

$-fv_{y}-fu_{x}=0$

so $v_{y}+u_{x}=0$ which is the quasistatic version of eq(1).


\subsubsection*{(b)}

if $u=-\psi_{y}$ and $v=\psi_{x}$

then $v_{y}+u_{x}=-\psi_{xy}+\psi_{xy}=0$ this equation is automatically
satisfied.

so $-f\psi_{x}+gh'_{x}=0$ and $-f\psi_{y}+gh'_{y}=0$ (10)

$\xi=(v_{x}-u_{y})=\psi_{xx}+\psi_{yy}=\nabla^{2}\psi$

Because of (10)

$\psi=\frac{g}{f}h'+Const$ (11)


\subsubsection*{(c)}

$v=(-\psi_{y},\psi_{x})$ and $\nabla\psi=(\psi_{x},\psi_{y})$

so $v\cdot\nabla\psi=0$

so they are perpendicular to each other.

From the form of $v$ and $\psi$ we know that $\nabla\psi$ is a
clockwise rotation of $v.$

So to the right of $v$ , $\psi$ always increases, this is no difference
for northern and southern hemisphere, this is just a natural outcome
of how you define your stream function. But if you link $\psi$ with
pressure, that is difference from Northern to Southern hemisphere
because the sign of $f$ is changed.

In northern hemisphere, $f$ is positive, so high $\psi$ means high
pressure, in southern hemisphere , high $\psi$ means low pressure.
So in northern hemisphere , high pressure is on the right of the flow
and in southern hemisphere, high pressure is on the left of the flow.


\subsubsection*{(d)}

$Q=\frac{\xi+f}{h}=\frac{\xi+f_{0}+\beta y}{h_{0}+h'(x,y,t)}=\frac{(\xi+f_{0}+\beta y)}{h_{0}(1+h'/h_{0})}=(1-\frac{h'}{h_{0}})(\frac{1}{h_{0}})(\xi+f_{0}+\beta y)=\frac{f_{0}}{h_{0}}+(\frac{\xi+\beta y-\frac{f_{0}h'}{h_{0}}}{h_{0}})$

and $\xi=\nabla^{2}\psi$ ,$h'=\frac{f_{0}}{g}\psi+C$ (we can take
$C=0$ for convenience)

Also there are some other small terms that are second order things:

$\frac{\xi h'}{h_{0}^{2}}$ that we neglect.


\subsubsection*{(e)}

Comparing the $q$ given here and my result in 4(d),

$(1/L_{d}^{2})=\frac{f_{0}^{2}}{h_{0}g}$ so $L_{d}=\frac{\sqrt{h_{0}g}}{f_{0}}$,
take in $h_{0}=4km$ and $f_{0}=10^{-4}s^{-1}$

$L_{d}=19798.98m/s^{2}$

Since $DQ/Dt=0$

$D(\frac{\nabla^{2}\psi+\beta y-(1/L_{d})^{2}\psi}{h_{0}})=\frac{1}{h_{0}}$

\begin{eqnarray*}
\frac{D(\nabla^{2}\psi+\beta y-(1/L_{d})^{2}\psi)}{Dt} & = & (\nabla\psi-(1/L_{d})^{2}\psi+\beta y)_{t}\\
 &  & -\psi_{y}(\nabla^{2}\psi-(1/L_{d})^{2}\psi)_{x}+\psi_{x}(\nabla^{2}\psi-(1/L_{d})^{2}\psi+\beta y)_{y}\\
 & = & (\nabla\psi-(1/L_{d})^{2}\psi+\beta y)_{t}+J(\psi,\nabla^{2}\psi)+\beta\psi_{x}
\end{eqnarray*}

\end{document}
