
% This LaTeX was auto-generated from MATLAB code.
% To make changes, update the MATLAB code and republish this document.

\documentclass{article}
\usepackage{graphicx}
\usepackage{color}

\sloppy
\definecolor{lightgray}{gray}{0.5}
\setlength{\parindent}{0pt}

\begin{document}

    
    \begin{verbatim}
function hw1_4
set(0,'defaulttextfontname','times','defaulttextfontsize',14);
set(0,'defaultaxesfontname','times','defaultaxesfontsize',14);

x = -20:0.1:20;
f = f_func(x);

figure(1)
plot(x,f,'ko-');
grid on; xlabel('x'); ylabel('f(x)');

epsilon = 1e-10;
x0 = -12:0.01:12;
itermax = 1000;

roots = zeros(length(x0),1);
flags = zeros(length(x0),1);
for i = 1:length(x0)
    [roots(i), flags(i)] = find_root(x0(i),epsilon,itermax);
end


figure(1)
hold on;
plot(roots,f_func(roots),'r*');
hold off;


figure(2)

subplot(121);
plot(x0,roots,'*'); xlabel('x0'); ylabel('roots');
subplot(122)
plot(x0,flags,'*'); xlabel('x0'); ylabel('find?');

end


function [root,flag] = find_root(x0,epsilon,itermax)
root = x0;
flag = false;   % whether you find a root
for i = 1:itermax
    f_curr = f_func(x0);
    f_grad_curr = f_grad_func(x0);

    if abs(f_grad_curr) < epsilon   % denominator is too small, stop
       break;
    end

    x1 = x0 - f_curr/f_grad_curr;

    if abs( x1-x0 ) < epsilon
        flag = true;     % find the root
        root = x0;
        break;
    end

    x0 = x1;
end
end

% value of the function at x
function val = f_func(x)
val = x.^2/10 + 6 * sin(x) + 2;
end

% value of the gradient at x
function val = f_grad_func(x)
val = x./5 + 6 * cos(x);
end
\end{verbatim}

\includegraphics [width=4in]{hw1_4_01.eps}

\includegraphics [width=4in]{hw1_4_02.eps}



\end{document}
    
