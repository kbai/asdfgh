%% LyX 2.0.5 created this file.  For more info, see http://www.lyx.org/.
%% Do not edit unless you really know what you are doing.
%% LyX 2.0.5 created this file.  For more info, see http://www.lyx.org/.
%% Do not edit unless you really know what you are doing.
\documentclass[12pt]{article}
\usepackage[margin=0.75in]{geometry} % see geometry.pdf on how to lay out the page. There's lots.
\usepackage{graphicx}
\usepackage{cleveref}
\usepackage{amsmath}
\usepackage{multirow}


\usepackage[latin9]{inputenc}
\usepackage{geometry}
\geometry{verbose}


\makeatletter
\@ifundefined{date}{}{\date{}}
\makeatother

%Fancy-header package to modify header/page numbering 
\usepackage{fancyhdr}
\pagestyle{fancy}
\lhead{\textbf{Ge/ESE 118}} %name of the course
\chead{\textbf{}} %topic of the homework set
\rhead{\textbf{Solution 1}} %number of the homework set
\lfoot{}
\cfoot{}
\rfoot{\thepage}
\begin{document}

\section{Problem 1 (graded by Kangchen) - 24 points + 1 free point}


\subsection*{(a) 4 points}

The model parameter vector is $\mathbf{m}=[P,d]^{T}$;


\subsection*{(b) 4 points}

$\boldsymbol{\nabla}u=[\frac{\partial u}{\partial P},\frac{\partial u}{\partial d}]^{T}=[\frac{d}{(x^{2}+y^{2}+d^{2})^{3/2}},\frac{P(x^{2}+y^{2}-2d^{2})}{(x^{2}+y^{2}+d^{2})^{5/2}}]^{T}$;


\subsection*{(c) 4 points}

$\boldsymbol{H}=\begin{bmatrix}\frac{\partial^{2}u}{\partial P^{2}} & \frac{\partial^{2}u}{\partial P\partial d}\\
\frac{\partial^{2}u}{\partial d\partial P} & \frac{\partial^{2}u}{\partial d^{2}}
\end{bmatrix}=\begin{bmatrix}0 & \frac{x^{2}+y^{2}-2d^{2}}{(x^{2}+y^{2}+d^{2})^{5/2}}\\
\frac{x^{2}+y^{2}-2d^{2}}{(x^{2}+y^{2}+d^{2})^{5/2}} & \frac{6Pd^{3}-9pdx^{2}-9pdy^{2}}{(x^{2}+y^{2}+d^{2})^{7/2}}
\end{bmatrix}$

Note that the $\frac{\partial^{2}u}{\partial P\partial d}=\frac{\partial^{2}u}{\partial d\partial P}$
given that the function $u(P,d)$ is smooth enough.


\subsection*{(d) 4 points}

No, since the two chambers may have different locations $(x,y,d)$
and different P.

The model parameter vector for two magma chambers should be $[P_{1},P_{2},d_{1},d_{2},x_{1},x_{2},y_{1},y_{2}]^{T}$

and the observed vertical displacement should be:

$u=\frac{P_{1}d_{1}}{((x-x_{1})^{2}+(y-y_{1})^{2}+d_{1}^{2})^{3/2}}+\frac{P_{2}d_{2}}{((x-x_{2})^{2}+(y-y_{2})^{2}+d_{2}^{2})^{3/2}}$

If there are two magma chambers , it is not easy to tell their locations
$(x,y)$ in the first place, so they need to be inverted from observations.
Also we need to have two different depth (d) parameters and P(pressure
related parameters) to describe the two magma chambers.

There are some other acceptable solutions that assume the two magma
chambers are at the same location and/or have the same P ...


\subsection*{(e) 4 points}

8 (Or other acceptable solutions)


\subsection*{(f) 4 points}

No, because there are 8 degrees of freedom in the model space while
only 6 degrees of freedom in the data space.

To map from a 8 dimensional model space to 6 dimensional data space
means that there will be different models being mapped to the same
point in data space.

This causes the inversion to have multiple solutions and trade off
between different parameters.

There are also other acceptable solutions. If your number of model
parameters is smaller than 6, then the solution may not exist. There
may not be a set of four parameters that fits the 6 data points exactly.
This is due to error in both model and measurements.


\section{Problem 2 (graded by Yiran) - 25 points}


\subsection*{(a) 6 points}

Let the matrix $\boldsymbol{A}=\left[\boldsymbol{x}_{1},\boldsymbol{x}_{2},\boldsymbol{x}_{3},\boldsymbol{x}_{4}\right]$,
then \\


\begin{eqnarray*}
\boldsymbol{A}= & \begin{bmatrix}1 & 0 & 0 & 1\\
0 & 1 & -1 & 1\\
0 & 1 & 1 & 1
\end{bmatrix}\\
\boldsymbol{A}^{T}\boldsymbol{A}= & \begin{bmatrix}1 & 0 & 0 & 1\\
0 & 2 & 0 & 2\\
0 & 0 & 2 & 0\\
1 & 2 & 0 & 3
\end{bmatrix}\\
= & \begin{bmatrix}\boldsymbol{x}_{1}^{T}\boldsymbol{x}_{1} & \boldsymbol{x}_{1}^{T}\boldsymbol{x}_{2} & \boldsymbol{x}_{1}^{T}\boldsymbol{x}_{3} & \boldsymbol{x}_{1}^{T}\boldsymbol{x}_{4}\\
 & \boldsymbol{x}_{2}^{T}\boldsymbol{x}_{2} & \boldsymbol{x}_{2}^{T}\boldsymbol{x}_{3} & \boldsymbol{x}_{2}^{T}\boldsymbol{x}_{4}\\
 & \multirow{2}{*}{\makebox[0pt]{\text{sym.}}} & \boldsymbol{x}_{3}^{T}\boldsymbol{x}_{3} & \boldsymbol{x}_{3}^{T}\boldsymbol{x}_{4}\\
 &  &  & \boldsymbol{x}_{4}^{T}\boldsymbol{x}_{4}
\end{bmatrix}
\end{eqnarray*}
Therefore, we see that $\boldsymbol{x}_{1}$, $\boldsymbol{x}_{2}$,
$\boldsymbol{x}_{3}$ are orthogonal to each other; and $\boldsymbol{x}_{3}$
and $\boldsymbol{x}_{4}$ are orthogonal.\\
 \\



\subsection*{(b) 6 points}

There are two independent vectors in $\left\{ \boldsymbol{x}_{1},\boldsymbol{x}_{2},\boldsymbol{x}_{4}\right\} $,
choosing any two from them, along with $\boldsymbol{x}_{3}$, which
is independent with all of them, form a basis. All the bases are:

\begin{center}
$\left\{ \boldsymbol{x}_{1},\boldsymbol{x}_{2},\boldsymbol{x}_{3}\right\} $,
$\left\{ \boldsymbol{x}_{1},\boldsymbol{x}_{4},\boldsymbol{x}_{3}\right\} $,
$\left\{ \boldsymbol{x}_{2},\boldsymbol{x}_{4},\boldsymbol{x}_{3}\right\} $ 
\par\end{center}


\subsection*{(c) 8 points}

By definition\\


\begin{eqnarray*}
\boldsymbol{AX}= & \boldsymbol{XD}
\end{eqnarray*}
where 
\begin{eqnarray*}
\boldsymbol{X}= & \left[\boldsymbol{x}_{1}\quad\boldsymbol{x}_{2}\quad\boldsymbol{x}_{3}\right]\\
= & \begin{bmatrix}1 & 0 & 0\\
0 & 1 & -1\\
0 & 1 & 1
\end{bmatrix}\\
\boldsymbol{D}= & \begin{bmatrix}\lambda_{1} & 0 & 0\\
0 & \lambda_{2} & 0\\
0 & 0 & \lambda_{3}
\end{bmatrix}\\
= & \begin{bmatrix}2 & 0 & 0\\
0 & 5 & 0\\
0 & 0 & 4
\end{bmatrix}\\
\end{eqnarray*}
\\
 Then\\


\begin{center}
$\boldsymbol{A}=\boldsymbol{XDX^{-1}}$ 
\par\end{center}

Since 
\begin{eqnarray*}
det(\boldsymbol{X})= & 2\\
\\
\boldsymbol{X}^{-1}= & 1/2\begin{bmatrix}2 & 0 & 0\\
0 & 1 & -1\\
0 & 1 & -1
\end{bmatrix}^{T}\\
= & \begin{bmatrix}1 & 0 & 0\\
0 & 0.5 & 0.5\\
0 & -0.5 & 0.5
\end{bmatrix}
\end{eqnarray*}


Then 
\begin{eqnarray*}
\boldsymbol{A}= & \begin{bmatrix}1 & 0 & 0\\
0 & 1 & -1\\
0 & 1 & 1
\end{bmatrix}\begin{bmatrix}2 & 0 & 0\\
0 & 5 & 0\\
0 & 0 & 4
\end{bmatrix}\begin{bmatrix}1 & 0 & 0\\
0 & 0.5 & 0.5\\
0 & -0.5 & 0.5
\end{bmatrix}= & \begin{bmatrix}2 & 0 & 0\\
0 & 4.5 & 0.5\\
0 & 0.5 & 4.5
\end{bmatrix}
\end{eqnarray*}
\\



\subsection*{(d) 5 points}

The vector is stretched by $\lambda_{1}$, $\lambda_{2}$, $\lambda_{3}$
along the corresponding eigenvector directions $\boldsymbol{x}_{1}$,
$\boldsymbol{x}_{2}$, $\boldsymbol{x}_{3}$. \\
 \\
 Check it with $\boldsymbol{x}=[1\quad2\quad3]^{T}$.\\
 \\
 Let $\boldsymbol{v}_{1}=\boldsymbol{X^{-1}x}=[1\quad2.5\quad0.5]^{T}$\\


\begin{eqnarray*}
\boldsymbol{x}= & \boldsymbol{XX^{-1}x}\\
= & \boldsymbol{Xv_{1}}\\
= & [\boldsymbol{x}_{1}\quad\boldsymbol{x}_{2}\quad\boldsymbol{x}_{3}]\begin{bmatrix}1\\
2.5\\
0.5
\end{bmatrix}\\
= & 1\cdot\boldsymbol{x}_{1}+2.5\cdot\boldsymbol{x}_{2}+0.5\cdot\boldsymbol{x}_{3}
\end{eqnarray*}
We see that $\boldsymbol{v}_{1}$ is the coordinate of $\boldsymbol{x}$
in the basis $\left\{ \boldsymbol{x}_{1},\boldsymbol{x}_{2},\boldsymbol{x}_{3}\right\} $.
\\
 Then 
\begin{eqnarray*}
\boldsymbol{Ax}= & \boldsymbol{XDX^{-1}x}\\
= & \boldsymbol{XD}\boldsymbol{v}_{1}\\
= & [\lambda_{1}\boldsymbol{x}_{1}\quad\lambda_{2}\boldsymbol{x}_{2}\quad\lambda_{3}\boldsymbol{x}_{3}]\boldsymbol{v}_{1}\\
= & \lambda_{1}\cdot1\cdot\boldsymbol{x}_{1}+\lambda_{2}\cdot2.5\cdot\boldsymbol{x}_{2}+\lambda_{3}\cdot0.5\cdot\boldsymbol{x}_{3}\\
\end{eqnarray*}
which we see is a stretch of $\boldsymbol{x}$ by $\lambda_{i}$ in
$\boldsymbol{x}_{i}$.\\
 This transformation gives \\
 $2\cdot[1\quad0\quad0]^{T}+12.5\cdot[0\quad1\quad1]^{T}+2\cdot[0\quad-1\quad1]^{T}=[2\quad10.5\quad14.5]^{T}$,
which is equal to the direct multiplication of $\boldsymbol{Ax}$.


\section{Problem 3 (graded by Kangchen) - 24 points + 1 free point }


\subsection*{(a) 6 points}

This implies $\boldsymbol{v}^{T}\boldsymbol{x}=0$ since $\boldsymbol{v}$
in $V$ means that $\boldsymbol{v}$ is orthogonal to $\boldsymbol{x}$.


\subsection*{(b) 6 points}

This is explaining that the set $V$ is a subspace of $\mathbf{R}^{n}$

$(a_{1}\boldsymbol{v}_{\mathbf{1}}+a_{2}\boldsymbol{v}_{2})^{T}x=a_{1}\boldsymbol{v}_{1}^{T}\boldsymbol{x}+a_{2}\boldsymbol{v}_{2}^{T}\boldsymbol{x}=a_{1}(0)+a_{2}(0)=0$

The first equality is by the distributive axiom of matrix multiplication.

This implies that $(a_{1}\boldsymbol{v}_{1}+a_{2}\boldsymbol{v}_{2})$
is also orthogonal to $\boldsymbol{x}$.

so $(a_{1}\boldsymbol{v}_{1}+a_{2}\boldsymbol{v}_{2})$ is in $V$


\subsection*{(c) 6 points}

Since $\boldsymbol{x}$ is an eigenvector of $\boldsymbol{A}$, we
assume $\boldsymbol{Ax}=k\boldsymbol{x}$

$(\boldsymbol{Ax}\mathit{)^{T}}x=\boldsymbol{v}^{T}\boldsymbol{A}^{T}\boldsymbol{x}=\boldsymbol{v}^{T}\boldsymbol{A}\boldsymbol{x}=\boldsymbol{v}^{T}(\boldsymbol{A}\boldsymbol{x})=\boldsymbol{v}^{T}k\boldsymbol{x}=k\boldsymbol{v}^{T}\boldsymbol{x}=0$

So $\boldsymbol{Av}$ is also orthogonal to $\boldsymbol{x}$,for
all $\boldsymbol{v}$ in $V$


\subsection*{(d) 6 points}

Given condition: For all $\boldsymbol{v}$ in $V$ ($V$ is not a
$\boldsymbol{0}$ space)if $\mathbf{\boldsymbol{Av}}$is also in $V$
,then one can find one eigenvector of $\mathbf{\boldsymbol{A}}$ in
subspace $V$.

$\mathbf{\boldsymbol{A}}$ is a $n\times n$ matrix.

First we can find one eigenvector $\boldsymbol{\mathbf{\mathbf{x}}_{1}}$
in $\mathbf{R}^{n}$and we get the subspace $V_{1}$, a subset of
$R^{n}$ that is orthogonal to $\mathbf{\boldsymbol{x}}_{1}.$

Then we can find another eigenvector $\mathbf{\boldsymbol{x}_{2}}$in
$V_{1}$ and get the subspace $V_{2}$, a subset of $V_{1}$ that
is orthogonal to $\mathbf{\boldsymbol{x}}_{1}$and $\mathbf{\boldsymbol{x}}_{2}$.

.........

Then we find $\mathbf{\boldsymbol{x}_{n-1}}$ in $V_{n-2}$ and get
the subspace $V_{n-1}$a subset of $V_{n-2}$ that is orthogonal to
$\mathbf{\boldsymbol{x}_{1},\boldsymbol{x}_{2},...\boldsymbol{x}_{n-1}}$.

$V_{n-1}$ will be a one dimensional subspace and we can find the
last eigenvector $\boldsymbol{x}_{n}$

Thus, we have found that all the eigenvectors are orthogonal to each
other.

This proof is enough for our purpose. We also attach the following
proof that is more strict in a mathematical sense. This is not necessary
to understand.

1) one can find one eigenvector $\mathbf{x}$ in $\mathbf{R^{n}}$.
Since $\mathbf{R}^{n}$ satisfies Given condition. This is trivial.

2) if a subspace $V$ satisfies Given condition, and we have an eigenvector
$\boldsymbol{x}$ in $V$. Then we prove that the set $U$ that contains
all vectors in $V$ that are orthogonal to $\boldsymbol{x}$ also
satisfies Given condition.

It is shown in (b) that $U$ forms a subspace .

for all $\mathbf{\mathbf{\boldsymbol{u}}}$ in $U,$ $\mathbf{\boldsymbol{Au}}$
is in $V$ since $U$ is a subset of $V$

we can write $\mathbf{\boldsymbol{Au}}=c\mathbf{\boldsymbol{x}}+\mathbf{\boldsymbol{w}},\mathbf{\boldsymbol{w}}$
is in $U$, multiply both sides by $\mathbf{x}^{T}$:

$\boldsymbol{x}^{T}\boldsymbol{A}\boldsymbol{u}=c\boldsymbol{x}^{T}\boldsymbol{x}+\boldsymbol{x}^{T}\boldsymbol{w}=c\boldsymbol{x}^{T}\boldsymbol{x}$

note that on the left hand side:$\boldsymbol{x}^{T}\boldsymbol{Au}=\boldsymbol{u}^{T}\boldsymbol{A}^{T}\boldsymbol{x}=\boldsymbol{u}^{T}\boldsymbol{A}\boldsymbol{x}=k\boldsymbol{u}^{T}\boldsymbol{x}=0$

So $c\boldsymbol{x}^{T}\boldsymbol{x}=c|\boldsymbol{\mathbf{x}}|^{2}=0,$which
implies that $c=0,$

So for all $\boldsymbol{u}$ in $U$ ,$\mathbf{\boldsymbol{Au}}$
is also in $U.$ So $U$ satisfies Given condition.

3) That means we can always find one eigenvector while the remaining
subspace still satisfies Given condition until it shrinks to $\{\boldsymbol{0}\}.$
Since each time the dimension of the space is decreased by 1, we can
find n orthogonal eigenvectors.


\section{Problem 4 (graded by Yiran) - 25 points}

(See atttached Matlab code and figures)\\
 Fig. 1 shows the function and the four roots.\\
 Fig. 2 shows the root found from each initial guess $x_{0}$. The
red lines mark the four roots. We also plot the function $f(x)$ in
black.\\
 Fig. 3 shows that for all the initial guesses, the method converges
within a certain number of iterations.\\
 \\
 There are four roots of this function. If the initial guess is close
enough to a given root, Newton's method will generally find this root
that is closest to the initial guess. \\
 If the initial guess is far away from the roots or close to an extreme
point, there is no guarantee that Newton's method will find the closest
root. The jump in x is controlled by the gradient at the current point.
If the current point is close to an extreme point (and note that there
are many extreme points of this function), the small gradient will
cause a large jump to the neighbor of another root. \\
 To find all the roots of the function, we need to try different initial
guesses. 
\end{document}
