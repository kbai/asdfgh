\documentclass[12pt]{article}
\usepackage[margin=1in]{geometry} % see geometry.pdf on how to lay out the page. There's lots.
\usepackage{graphicx}
\usepackage{cleveref}
\usepackage{amsmath}
\usepackage{multirow}

\title{Homework 1, Problem 2 \& 4}
\author{Yiran Ma}
\date{}

%%% BEGIN DOCUMENT
\begin{document}
\maketitle

% Problem 2
\section{Problem 2}
(a) Let the matrix $ A= \left[   \boldsymbol{x}_{1}, \boldsymbol{x}_{2}, \boldsymbol{x}_{3}, \boldsymbol{x}_{4}    \right] $, then \\
\begin{eqnarray*}
A =& \begin{bmatrix}
1&  0&  0&    1\\
0&  1&  -1&   1\\
0&  1&   1&    1\\
\end{bmatrix} \\
A^{T}A =& \begin{bmatrix}
1&  0&  0&    1\\
0&  2&  0&    2\\
0&  0&   2&    0\\
1&  2&   0&    3\\
\end{bmatrix} \\
=&\begin{bmatrix}
\boldsymbol{x}_{1}^{T}\boldsymbol{x}_{1}&  \boldsymbol{x}_{1}^{T}\boldsymbol{x}_{2}&    \boldsymbol{x}_{1}^{T}\boldsymbol{x}_{3}&      \boldsymbol{x}_{1}^{T}\boldsymbol{x}_{4} \\  
 &       \boldsymbol{x}_{2}^{T}\boldsymbol{x}_{2}&       \boldsymbol{x}_{2}^{T}\boldsymbol{x}_{3}&        \boldsymbol{x}_{2}^{T}\boldsymbol{x}_{4}     \\
 &  \multirow{2}{*}{\makebox[0pt]{\text{sym.}}}        &          \boldsymbol{x}_{3}^{T}\boldsymbol{x}_{3}&         \boldsymbol{x}_{3}^{T}\boldsymbol{x}_{4}     \\
 &         &         &     \boldsymbol{x}_{4}^{T}\boldsymbol{x}_{4}\\       
\end{bmatrix}
\end{eqnarray*}
Therefore, we see that $\boldsymbol{x}_{1}$, $\boldsymbol{x}_{2}$, $\boldsymbol{x}_{3}$ are orthogonal to each other; and $\boldsymbol{x}_{3}$ and $\boldsymbol{x}_{4}$ are orthogonal.\\
\\
(b) There are two independent vectors in $\left \{ \boldsymbol{x}_{1}, \boldsymbol{x}_{2}, \boldsymbol{x}_{4}  \right \}$, choosing any two from them, along with $\boldsymbol{x}_{3}$, which is independent with all of them, form a basis. All the bases are:
\begin{center}
$\left \{ \boldsymbol{x}_{1}, \boldsymbol{x}_{2}, \boldsymbol{x}_{3}  \right \}$, $\left \{ \boldsymbol{x}_{1}, \boldsymbol{x}_{4}, \boldsymbol{x}_{3}  \right \}$, $\left \{ \boldsymbol{x}_{2}, \boldsymbol{x}_{4}, \boldsymbol{x}_{3}  \right \}$
\end{center}
(c)
\begin{eqnarray*}
A\left[ \frac{1}{\lambda_{1}}\boldsymbol{x}_{1} \quad   \frac{1}{\lambda_{2}}\boldsymbol{x}_{2}  \quad   \frac{1}{\lambda_{3}}\boldsymbol{x}_{3} \right]
= \left[\boldsymbol{x}_{1} \quad  \boldsymbol{x}_{2}  \quad \boldsymbol{x}_{3} \right] 
\end{eqnarray*}
Let
\begin{eqnarray*}
B =& \left[ \frac{1}{\lambda_{1}}\boldsymbol{x}_{1} \quad   \frac{1}{\lambda_{2}}\boldsymbol{x}_{2}  \quad   \frac{1}{\lambda_{3}}\boldsymbol{x}_{3} \right]\\
=
& \begin{bmatrix}
\frac{1}{2}&  0&    0\\
0&  \frac{1}{5}&   -\frac{1}{4}\\
0&  \frac{1}{5}&    \frac{1}{4}\\
\end{bmatrix}\\
C=& \left[   \boldsymbol{x}_{1} \quad \boldsymbol{x}_{2} \quad \boldsymbol{x}_{3}    \right] \\
=
&\begin{bmatrix}
1&  0&  0\\
0&  1&  -1\\
0&  1&   1\\
\end{bmatrix}
\end{eqnarray*}\\
Then\\
\begin{center}
$A = CB^{-1}$
\end{center}
Since
\begin{eqnarray*}
det(B)=&1/20\\
\\
B^{-1}=&20
\begin{bmatrix}
\frac{1}{10}&  0&    0\\
0&  \frac{1}{8}&   -\frac{1}{10}\\
0&  \frac{1}{8}&    \frac{1}{10}\\
\end{bmatrix}^{T}\\
=
&\begin{bmatrix}
2&  0&  0\\
0&  2.5&  2.5\\
0&  -2&   2\\
\end{bmatrix}
\end{eqnarray*}
Then
\begin{eqnarray*}
A=
&\begin{bmatrix}
1&  0&  0\\
0&  1&  -1\\
0&  1&   1\\
\end{bmatrix}
\begin{bmatrix}
2&  0&  0\\
0&  2.5&  2.5\\
0&  -2&   2\\
\end{bmatrix}
=
&\begin{bmatrix}
2&  0&  0\\
0&  4.5&  0.5\\
0&  0.5&   4.5\\
\end{bmatrix}
\end{eqnarray*}\\
(c)\\
Since A is an symmetric matrix, we have $A=Q\Lambda Q^{-1}$, where 
\begin{eqnarray*}
Q=&\left[ \boldsymbol{\tilde{x}}_{1}\quad  \boldsymbol{\tilde{x}}_{2}\quad \boldsymbol{\tilde{x}}_{3}\right] 
\end{eqnarray*}
$\boldsymbol{\tilde{x}}_{i}$ is the normalized $\boldsymbol{x}_{i}$.

In this problem, 
\begin{eqnarray*}
Q=\begin{bmatrix}
1&  0&  0\\
0&  \frac{\sqrt{2}}{2}&  -\frac{\sqrt{2}}{2}\\
0&  \frac{\sqrt{2}}{2}&   \frac{\sqrt{2}}{2}\\
\end{bmatrix}
\end{eqnarray*}
It's a 45-degree axial rotation along $\boldsymbol{e}_{1}$.
\begin{eqnarray*}
\Lambda=
&\begin{bmatrix}
2 &  0&  0\\
0&  5&  0\\
0&  0&   4\\
\end{bmatrix}
\end{eqnarray*}
It's stretching along three standard basis directions.\\
Therefore, the transformation includes: (1) A -45 degree axial rotation along $\boldsymbol{e}_{1}$; (2) 2, 5 and 4 times stretching along the three standard basis directions; (3) A 45 degree axial rotation back.\\
Check it with $ \boldsymbol{x} = [ 1\quad 2\quad 3 ]^{T} $.
\begin{eqnarray*}
v_{1} =& Q^{-1}x =& [1\quad 3.5355 \quad 0.7071]^{T} \\
v_{2} =& \Lambda v_{1} =& [2\quad 17.6777 \quad 2.8284]^{T}\\
v_{3} =& Qv_{2} =& [2\quad 10.5\quad 14.5] = Ax
\end{eqnarray*}

\newpage
% Problem 4
\section{Problem 4}
There is no evidence that Newton's method will find the root closest to the initial guess. Because the "jump" in x is largely controlled by the gradient at current point, a small gradient will cause a large jump to the neighbor of another root.


\end{document}
